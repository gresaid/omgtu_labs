%%%%%%%%%%%%%%%%%%%%%%%%%%%%%%%%%%%%%%%%%%%%%%%%%%%%%%
%%%% Файл упрощённых настроек шаблона диссертации %%%%
%%%%%%%%%%%%%%%%%%%%%%%%%%%%%%%%%%%%%%%%%%%%%%%%%%%%%%

%%% Инициализирование переменных, не трогать!  %%%
\newcounter{intvl}
\newcounter{otstup}
\newcounter{contnumeq}
\newcounter{contnumfig}
\newcounter{contnumtab}
\newcounter{pgnum}
\newcounter{chapstyle}
\newcounter{headingdelim}
\newcounter{headingalign}
\newcounter{headingsize}
%%%%%%%%%%%%%%%%%%%%%%%%%%%%%%%%%%%%%%%%%%%%%%%%%%%%%%

%%% Область упрощённого управления оформлением по новому ГОСТу %%%

%% Интервал между заголовками и между заголовком и текстом %%
% По новому ГОСТу заголовки отделяются пустой строкой сверху и снизу
\setcounter{intvl}{1}               % Коэффициент = 1 (одна пустая строка)

%% Отступы у заголовков в тексте %%
% По новому ГОСТу:
% - Основные структурные единицы: БЕЗ абзацного отступа, по центру
% - Разделы/подразделы: С абзацным отступом
% - Пункты: С абзацным отступом, без жира
\setcounter{otstup}{1}              % 1 --- абзацный отступ для разделов/подразделов

%% Нумерация формул, таблиц и рисунков %%
% Нумерация формул
\setcounter{contnumeq}{0}   % 0 --- пораздельно (во введении подряд,
%       без номера раздела);
% 1 --- сквозная нумерация по всей диссертации
% Нумерация рисунков
\setcounter{contnumfig}{1}  % 1 --- сквозная нумерация по всей диссертации (обычно так по ГОСТу)
% Нумерация таблиц
\setcounter{contnumtab}{1}  % 1 --- сквозная нумерация по всей диссертации (обычно так по ГОСТу)

%% Оглавление %%
\setcounter{pgnum}{0}       % 1 --- Стр. над номерами страниц (как требует ГОСТ)
\settocdepth{subsubsection} % до пунктов включительно (по новому ГОСТу)
\setsecnumdepth{subsubsection} % до пунктов включительно

%% Текст и форматирование заголовков по новому ГОСТу %%
\setcounter{chapstyle}{0}     % 0 --- основные структурные единицы без "Глава"
\setcounter{headingdelim}{0}  % 0 --- БЕЗ точек в конце номеров (по новому ГОСТу)

%% Выравнивание заголовков в тексте %%
\setcounter{headingalign}{1}  % 1 --- разделы/подразделы по левому краю с абзацным отступом

%% Размеры заголовков в тексте %%
\setcounter{headingsize}{0}   % 0 --- все заголовки 14 пт (по новому ГОСТу)

%% Подпись таблиц по новому ГОСТу %%

% Смещение строк подписи после первой строки
\newcommand{\tabindent}{0cm}

% Тип форматирования заголовка таблицы:
% plain --- название и текст в одной строке (используем по новому ГОСТу)
\newcommand{\tabformat}{plain}

%%% Настройки форматирования таблицы `plain` по новому ГОСТу

% Выравнивание по центру подписи, состоящей из одной строки:
% false --- НЕ выравнивать по центру (по левому краю таблицы)
\newcommand{\tabsinglecenter}{false}

% Выравнивание подписи таблиц:
% raggedright --- выравнивать по левому краю (как требует новый ГОСТ)
\newcommand{\tabjust}{raggedright}

% Разделитель записи «Таблица #» и названия таблицы
% По новому ГОСТу: Таблица <номер> – Название таблицы
\newcommand{\tablabelsep}{~--~}

%%% Настройки форматирования таблицы `split` (если понадобится)

% Положение названия таблицы:
% \raggedright --- выравнивать по левому краю
\newcommand{\splitformatlabel}{\raggedright}

% Положение текста подписи:
% \raggedright --- выравнивать по левому краю
\newcommand{\splitformattext}{\raggedright}

%% Подпись рисунков %%
% Разделитель записи «Рисунок #» и названия рисунка
% По ГОСТу используется тире
\newcommand{\figlabelsep}{~--~}

%%% Цвета гиперссылок %%%
% Можно использовать черный цвет для печатной версии
\definecolor{linkcolor}{rgb}{0,0,0}   % черный для ссылок
\definecolor{citecolor}{rgb}{0,0,0}   % черный для цитат
\definecolor{urlcolor}{rgb}{0,0,0}    % черный для URL

% Или цветные для электронной версии (раскомментировать при необходимости)
%\definecolor{linkcolor}{rgb}{0.9,0,0}
%\definecolor{citecolor}{rgb}{0,0.6,0}
%\definecolor{urlcolor}{rgb}{0,0,1}
