%%% Переопределение именований, если иначе не сработает %%%
%\gappto\captionsrussian{
%    \renewcommand{\chaptername}{Глава}
%    \renewcommand{\appendixname}{Приложение} % (ГОСТ Р 7.0.11-2011, 5.7)
%}

%%% Изображения %%%
\graphicspath{{images/}{Dissertation/images/}}         % Пути к изображениям

%%% Интервалы %%%
%% По новому ГОСТу требуется междустрочный интервал 1.5
%% Реализация средствами класса (на основе setspace) ближе к типографской классике.
%% И правит сразу и в таблицах (если со звёздочкой)
%\DoubleSpacing*     % Двойной интервал
\OnehalfSpacing*    % Полуторный интервал (1.5)
%\setSpacing{1.42}   % Полуторный интервал, подобный Ворду (возможно, стоит включать вместе с предыдущей строкой)

%%% Макет страницы %%%
% Выставляем значения полей по новому ГОСТу: верх и низ по 2см, лево 3см, право 1.5см
\geometry{a4paper, top=2cm, bottom=2cm, left=3cm, right=1.5cm, nofoot, nomarginpar} %, heightrounded, showframe
\setlength{\topskip}{0pt}   %размер дополнительного верхнего поля
\setlength{\footskip}{12.3pt} % снимет warning, согласно https://tex.stackexchange.com/a/334346

%%% Выравнивание и переносы %%%
%% http://tex.stackexchange.com/questions/241343/what-is-the-meaning-of-fussy-sloppy-emergencystretch-tolerance-hbadness
%% http://www.latex-community.org/forum/viewtopic.php?p=70342#p70342
\tolerance 1414
\hbadness 1414
\emergencystretch 1.5em % В случае проблем регулировать в первую очередь
\hfuzz 0.3pt
\vfuzz \hfuzz
%\raggedbottom
%\sloppy                 % Избавляемся от переполнений
\clubpenalty=10000      % Запрещаем разрыв страницы после первой строки абзаца
\widowpenalty=10000     % Запрещаем разрыв страницы после последней строки абзаца
\brokenpenalty=4991     % Ограничение на разрыв страницы, если строка заканчивается переносом

%%% Абзацный отступ %%%
\setlength{\parindent}{1.25cm} % Абзацный отступ 1.25 см
\setlength{\parskip}{0pt} % Убираем интервалы между абзацами

%%% Переопределение абзацного отступа из styles.tex %%%
\AtBeginDocument{%
    \setlength{\parindent}{1.25cm}     % Абзацный отступ 1.25см по новому ГОСТу (было 2.5em)
    \setlength{\parskip}{0pt}          % Убираем пустые интервалы до и после абзацев
}

%%% Блок управления параметрами для выравнивания заголовков в тексте %%%
\newlength{\otstuplen}
\setlength{\otstuplen}{\parindent} % Абзацный отступ для заголовков разделов/подразделов

% Заголовки основных структурных единиц (аннотация, содержание, введение, заключение и т.д.)
% CAPS LOCK, без жира, по центру, без абзацного отступа
\newcommand{\structuralheading}[1]{%
    \begin{center}%
    \MakeUppercase{#1}%
    \end{center}%
}

%%% Оглавление %%%
\renewcommand{\cftchapterdotsep}{\cftdotsep}                % отбивка точками до номера страницы начала главы/раздела

%% По новому ГОСТу нумерация БЕЗ точек в конце
\renewcommand\cftchapteraftersnum{\space}       % убираем точку после номера раздела в оглавлении
\renewcommand\cftsectionaftersnum{\space}       % убираем точку после номера подраздела в оглавлении
\renewcommand\cftsubsectionaftersnum{\space}    % убираем точку после номера подподраздела в оглавлении
\renewcommand\cftsubsubsectionaftersnum{\space} % убираем точку после номера подподподраздела в оглавлении

% Отступы в содержании согласно ГОСТу
\setlength{\cftsectionindent}{2em}      % подразделы - отступ 2 знака от разделов
\setlength{\cftsubsectionindent}{4em}   % пункты - отступ 4 знака от разделов

%% Переносить слова в заголовке не допускается. Заголовки в оглавлении должны точно повторять заголовки в тексте.
\setrmarg{2.55em plus1fil}                             %To have the (sectional) titles in the ToC, etc., typeset ragged right with no hyphenation
\renewcommand{\cftchapterpagefont}{\normalfont}        % нежирные номера страниц у глав в оглавлении
\renewcommand{\cftchapterleader}{\cftdotfill{\cftchapterdotsep}}% нежирные точки до номеров страниц у глав в оглавлении

% Настройки нумерации без точек в тексте
\AfterEndPreamble{% без этого polyglossia сама всё переопределяет
    \setsecnumformat{\csname the#1\endcsname\space} % убираем точки в нумерации заголовков
}

\renewcommand*{\cftappendixname}{\appendixname\space} % Слово Приложение в оглавлении

%%% Колонтитулы %%%
% Порядковый номер страницы печатают на середине верхнего поля страницы
\makeevenhead{plain}{}{}{}
\makeoddhead{plain}{}{}{}
\makeevenfoot{plain}{}{\rmfamily\thepage}{}
\makeoddfoot{plain}{}{\rmfamily\thepage}{}
\pagestyle{plain}

%%% добавить Стр. над номерами страниц в оглавлении
%%% http://tex.stackexchange.com/a/306950
\newif\ifendTOC

\newcommand*{\tocheader}{
    \ifnumequal{\value{pgnum}}{1}{%
        \ifendTOC\else\hbox to \linewidth%
        {\noindent{}~\hfill{Стр.}}\par%
        \ifnumless{\value{page}}{3}{}{%
            \vspace{0.5\onelineskip}
        }
        \afterpage{\tocheader}
        \fi%
    }{}%
}%

%%% Оформление заголовков глав, разделов, подразделов %%%
%% Базовый размер шрифта 14 пт согласно новому ГОСТу
\newcommand{\basegostsectionfont}{\fontsize{14pt}{21pt}\selectfont\bfseries} % 14 пт с интервалом 1.5
\newcommand{\basegostparagraphfont}{\fontsize{14pt}{21pt}\selectfont} % для пунктов - без жира

% Стиль глав (основных структурных единиц)
\makechapterstyle{thesisgost}{%
    \chapterstyle{default}
    \setlength{\beforechapskip}{0pt}
    \setlength{\midchapskip}{0pt}
    \setlength{\afterchapskip}{\onelineskip} % отделяется пустой строкой
    % Основные структурные единицы: CAPS LOCK, без жира, по центру
    \renewcommand*{\chapnamefont}{\fontsize{14pt}{21pt}\selectfont}
    \renewcommand*{\chapnumfont}{\fontsize{14pt}{21pt}\selectfont}
    \renewcommand*{\chaptitlefont}{\fontsize{14pt}{21pt}\selectfont}
    \renewcommand*{\chapterheadstart}{}
    \renewcommand*{\afterchapternum}{}     % без точки после номера
    \renewcommand*{\printchapternum}{\centering}
    \renewcommand*{\printchaptername}{}
    \renewcommand*{\printchapternonum}{\centering}
    \renewcommand*{\printchaptertitle}[1]{\centering\MakeUppercase{##1}}
}

\makeatletter
\makechapterstyle{thesisgostchapname}{%
    \chapterstyle{thesisgost}
    \renewcommand*{\printchapternum}{\chapnumfont \thechapter}
    \renewcommand*{\printchaptername}{\centering\chapnamefont \MakeUppercase{\@chapapp}} %
}
\makeatother

\chapterstyle{thesisgost}

% Разделы: абзацный отступ, жир, выравнивание по ширине/левому краю
\setsecheadstyle{\basegostsectionfont}
\setsecindent{\otstuplen} % абзацный отступ

% Подразделы: абзацный отступ, жир, выравнивание по ширине/левому краю
\setsubsecheadstyle{\basegostsectionfont}
\setsubsecindent{\otstuplen} % абзацный отступ

% Пункты: абзацный отступ, БЕЗ жира, выравнивание по ширине/левому краю
\setsubsubsecheadstyle{\basegostparagraphfont} % без жира для пунктов
\setsubsubsecindent{\otstuplen} % абзацный отступ

\sethangfrom{\noindent #1} %все заголовки подразделов с абзацным отступом

\ifnumequal{\value{chapstyle}}{1}{%
    \chapterstyle{thesisgostchapname}
    \renewcommand*{\cftchaptername}{\chaptername\space} % будет вписано слово Глава перед каждым номером раздела в оглавлении
}{}%

%%% Интервалы между заголовками
% Заголовки отделяют от текста пустой строкой сверху и снизу
\setbeforesecskip{\onelineskip}
\setaftersecskip{\onelineskip}
\setbeforesubsecskip{\onelineskip}
\setaftersubsecskip{\onelineskip}
\setbeforesubsubsecskip{\onelineskip}
\setaftersubsubsecskip{\onelineskip}

%%% Убираем дополнительные отступы %%%
% Убираем отступы до и после различных элементов в обычном тексте
\setlength{\abovedisplayskip}{\baselineskip}      % пустая строка перед формулами
\setlength{\belowdisplayskip}{\baselineskip}      % пустая строка после формул
\setlength{\abovedisplayshortskip}{\baselineskip} % пустая строка перед короткими формулами
\setlength{\belowdisplayshortskip}{\baselineskip} % пустая строка после коротких формул

% Настройки для таблиц и рисунков - пустые строки до и после
\setlength{\floatsep}{\baselineskip}        % расстояние между соседними флоатами
\setlength{\textfloatsep}{\baselineskip}    % расстояние между флоатом и текстом
\setlength{\intextsep}{\baselineskip}       % расстояние вокруг флоатов внутри текста

%%% Вертикальные интервалы глав (\chapter) в оглавлении как и у заголовков
% раскомментировать следующие 2
% \setlength{\cftbeforechapterskip}{0pt plus 0pt}   % ИЛИ эти 2 строки из учебника
% \renewcommand*{\insertchapterspace}{}
% или эту
% \renewcommand*{\cftbeforechapterskip}{0em}

%%% Блок дополнительного управления размерами заголовков
\ifnumequal{\value{headingsize}}{1}{% Если нужны пропорциональные заголовки
    \renewcommand{\basegostsectionfont}{\large\bfseries}
    \renewcommand*{\chapnamefont}{\Large}
    \renewcommand*{\chapnumfont}{\Large}
    \renewcommand*{\chaptitlefont}{\Large}
}{}

%%% Счётчики %%%

%% Упрощённые настройки шаблона диссертации: нумерация формул, таблиц, рисунков
\ifnumequal{\value{contnumeq}}{1}{%
    \counterwithout{equation}{chapter} % Убираем связанность номера формулы с номером главы/раздела
}{}
\ifnumequal{\value{contnumfig}}{1}{%
    \counterwithout{figure}{chapter}   % Убираем связанность номера рисунка с номером главы/раздела
}{}
\ifnumequal{\value{contnumtab}}{1}{%
    \counterwithout{table}{chapter}    % Убираем связанность номера таблицы с номером главы/раздела
}{}

\AfterEndPreamble{
%% регистрируем счётчики в системе totcounter
    \regtotcounter{totalcount@figure}
    \regtotcounter{totalcount@table}       % Если иным способом поставить в преамбуле то ошибка в числе таблиц
    \regtotcounter{TotPages}               % Если иным способом поставить в преамбуле то ошибка в числе страниц
    \newtotcounter{totalappendix}
    \newtotcounter{totalchapter}
}

%%% Дополнительные команды для оформления по новому ГОСТу %%%

% Команда для оформления основных структурных единиц
\newcommand{\mainsection}[1]{%
    \chapter*{\MakeUppercase{#1}}%
    \addcontentsline{toc}{chapter}{\MakeUppercase{#1}}%
}

% Команда для оформления приложений
\newcommand{\appendixsection}[1]{%
    \chapter*{\MakeUppercase{ПРИЛОЖЕНИЕ} \arabic{totalappendix} \\ \MakeUppercase{#1}}%
    \addcontentsline{toc}{chapter}{\MakeUppercase{ПРИЛОЖЕНИЕ} \arabic{totalappendix}. \MakeUppercase{#1}}%
    \stepcounter{totalappendix}%
}
