\chapter*{ЗАКЛЮЧЕНИЕ}     % Заголовок
\addcontentsline{toc}{chapter}{ЗАКЛЮЧЕНИЕ}  % Добавляем его в оглавление

%% Согласно ГОСТ Р 7.0.11-2011:
%% 5.3.3 В заключении диссертации излагают итоги выполненного исследования, рекомендации, перспективы дальнейшей разработки темы.
%% 9.2.3 В заключении автореферата диссертации излагают итоги данного исследования, рекомендации и перспективы дальнейшей разработки темы.
%% Поэтому имеет смысл сделать эту часть общей и загрузить из одного файла в автореферат и в диссертацию:

В ходе выполнения данной научно-исследовательской работы была проведена комплексная разработка и анализ микросервисной архитектуры для информационно-аналитической системы <<Православный ландшафт таежной Сибири>>. Исследование охватило ключевые аспекты современного проектирования распределенных систем с особым акцентом на решении проблем масштабируемости, отказоустойчивости и безопасности.
%% Согласно ГОСТ Р 7.0.11-2011:
%% 5.3.3 В заключении диссертации излагают итоги выполненного исследования, рекомендации, перспективы дальнейшей разработки темы.
%% 9.2.3 В заключении автореферата диссертации излагают итоги данного исследования, рекомендации и перспективы дальнейшей разработки темы.
\begin{enumerate}
  \item Разработана комплексная микросервисная архитектура, основанная на принципах декомпозиции предметной области и обеспечивающая независимое развертывание и масштабирование компонентов системы.
  \item Реализовано маршрутизирование на основе Spring Cloud Gateway,.
  \item Разработана и внедрена система централизованного управления конфигурациями на базе Spring Cloud Config
\end{enumerate}


Результаты проведенного исследования демонстрируют эффективность применения современных технологий микросервисной архитектуры для создания информационно-аналитических систем в области гуманитарных наук. Разработанное решение обеспечивает необходимый уровень масштабируемости, безопасности и производительности для обработки больших объемов исторических данных.

В заключение автор выражает благодарность и большую признательность проекту, существующему при поддержке Российского Научного
Фонда Грант номер «23-78-10119».
