
{\actuality}Научная значимость и актуальность проблемы обусловлены отсутствием в
отечественных и зарубежных исследованиях системного подхода к проектированию исследовательских инфраструктур в гуманитарных науках и их интеграции
в цифровые научные коммуникации. Развитие гуманитарных наук демонстрирует необходимость решения этой проблемы по следующим причинам. Во-первых,
расширяется спектр цифровых инструментов, требующих разработки методик,
позволяющих применять их к различным историческим источникам в рамках
научно обоснованных моделей создаваемых ресурсов. Во-вторых, возрастает значение сетевых взаимодействий и междисциплинарных проектов, что требует
детального понимания механизмов интеграции в научные коммуникации.
Современная историческая наука нуждается в моделях, способствующих
развитию междисциплинарных и сетевых взаимодействий. Эту задачу могут решить исторические информационные ресурсы, обеспечивающие интеграцию в
актуальные научные коммуникации и устойчивость в условиях динамичных изменений.
Разрабатываемая аналитическая модель исторического информационного
ресурса «Православный ландшафт таежной Сибири: акторы, институты, сети»
предоставляет возможность апробировать теоретические и методологические
подходы в области информационной инфраструктуры.

% {\progress}
% Этот раздел должен быть отдельным структурным элементом по
% ГОСТ, но он, как правило, включается в описание актуальности
% темы. Нужен он отдельным структурынм элемементом или нет ---
% смотрите другие диссертации вашего совета, скорее всего не нужен.

{\aim} данной работы является создание научно обоснованной аналитической модели исторического информационного ресурса «Православный ландшафт
таежной Сибири: акторы, институты, сети», включающей выявление зависимостей между социальными, культурными и природными факторами формирования
поселенческой сети методами математической статистики.

Для~достижения поставленной цели необходимо было решить следующие {\tasks}:
\begin{enumerate}[beginpenalty=10000] % https://tex.stackexchange.com/a/476052/104425
  \item Исследовать экосистему программных средств, используемых в предметной области тематики НИР.
  \item Проведение аналитического обзора существующих решений (программных продуктов) в изучаемой предметной области.
  \item Разработать серверную часть программного продукта для проекта.
\end{enumerate}


{\novelty}
\begin{enumerate}[beginpenalty=10000] % https://tex.stackexchange.com/a/476052/104425
  \item Впервые системно рассматривается комплекс вопросов, связанных с
  историческими информационными ресурсами по истории Сибири, развитием информационной инфраструктуры для социогуманитарных наук.
  \item Впервые будет сформулирована модель оценки степени интеграции
  исторических информационных ресурсов в научные коммуникации и
  исследовательскую инфраструктуру, описана карта оценки рисков сохранности информации по истории Сибири в цифровой среде.
  \item Впервые исторические процессы в регионе, корпус источников исторической информации тематически связанный с ее конфессиональным
  аспектом изучается на основе фундаментального математического аппарата.
\end{enumerate}
